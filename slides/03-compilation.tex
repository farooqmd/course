\documentclass{slides}

\lstset{language=C++}

\usepackage{xspace}

\newcommand{\ie}{\textit{i.\thinspace e.}\xspace}
\newcommand{\eg}{\textit{e.\thinspace g.}\xspace}

\begin{document}

\graphicspath{{figures/}}

\title[The Compilation Process]{\Large The Compilation Process}

\author[O. Lenz]{Olaf Lenz} 
\institute{Institut für Computerphysik\\Universit\"at Stuttgart}
\date{February 18-22, 2013}

\setbeamertemplate{footline}{}
\begin{frame}
  \titlepage
\end {frame}
\setbeamertemplate{footline}[icp]

\begin{frame}[fragile]
  \frametitle{Separate Compilation}
  \begin{itemize}
  \item So far, all programs were a single file
  \item Large software projects consist of thousands of lines of
    code
  \item Problem:
    \begin{itemize}
    \item Compilation may take very long
    \item It is not feasible to have all code in a single file
    \end{itemize}
  \item Better: split the project into several files, \emph{compile}
    them one by one and \emph{link} them together at the end
  \item Problem: To be able to compile a function, some things about
    all functions, variables and types that occur must be known
    \begin{lstlisting}
int main() {
  cout << "Hello, World!" << endl;
}    
\end{lstlisting}
    \item What is \lstinline!cout!, \lstinline!<<! and \lstinline!endl!?
  \end{itemize}
  
\end{frame}

\begin{frame}
  \frametitle{Declarations vs. Definitions}
  \begin{itemize}
  \item All functions and variables (\emph{symbols}) in C++ can be
    \emph{declared} and \emph{defined}
  \item \emph{Declaration} of symbols introduce names to the compiler:
    ``This function or this variable exists somewhere and here is what
    it should look like.''
  \item \emph{Definitions} tell the compiler: ``Make this variable
    or function here.''
  \item Declarations can be repeated, definitions must be unique
  \item A \emph{header file} (\eg \texttt{.hpp}) should only contain
    declarations (it is included in other header and source files)
  \item A \emph{source file} (\eg \texttt{.cpp}) contains definitions
    as well as declarations
  \item All symbols declared in a header file should be defined in
    exactly one source file
  \end{itemize}
\end{frame}

\begin{frame}[fragile]
  \frametitle{Function Declarations and Definitions}
  \begin{itemize}
  \item Declaration
    \begin{lstlisting}
int func(int length, int width);
    \end{lstlisting}
  \item Provides the signature (name, parameters and return type) of a
    function
  \item The names are ignored by the compiler
  \item Definition
    \begin{lstlisting}
int func(int length, int width) {
  // function code
  .
  .
}
    \end{lstlisting}
  \end{itemize}
\end{frame}

\begin{frame}[fragile]
  \frametitle{Variable Declarations and Definitions}
  \begin{itemize}
  \item Only interesting for \emph{global variables}
  \item Declaration
    \begin{lstlisting}
extern int a;
    \end{lstlisting}
  \item Definition
    \begin{lstlisting}
int a;
    \end{lstlisting}
  \end{itemize}
\end{frame}

\begin{frame}[fragile]
  \frametitle{Stage 1: Preprocessor inclusions}

    \begin{itemize}
    \item Pure text replacement
    \item \lstinline!#include FILE! is replaced by the contents of the
      \emph{header file} FILE
    \item The files have to be in the \emph{include path}
      \begin{itemize}
      \item The default include path contains \eg \texttt{/usr/include}
      \item It can be extended in the compiler call by \texttt{-I
          <path>}
      \item \lstinline!#include <file>! for system headers
      \item \lstinline!#include "file.hpp"! for own headers
      \end{itemize}
    \item Outputs a single file that should be a self-contained
      C++-program
    \item \lstinline!g++ -E FILE!: Preprocess FILE, output
      \lstinline!.c!-file.
    \end{itemize}
\end{frame}

\begin{frame}[fragile]
  \frametitle{Stage 1: Preprocessor Macros}
  \begin{itemize}
  \item \lstinline!#define NAME REPLACEMENT! defines a
    \emph{macro} with the given NAME
  \item Whenever the macro name turns up in the file, it is replaced
    by the REPLACEMENT
  \item This \emph{was} used for constants (\eg \lstinline!M_PI!)
  \item Preprocessor Conditionals
    \begin{lstlisting}
#ifdef MACRO
.
.
#else
.
.
#endif
    \end{lstlisting}
  \item Nowadays mostly used for compile time guards
  \end{itemize}
\end{frame}

\begin{frame}[fragile]
  \frametitle{Stage 1: Preprocessor Compile Time Guards}
    \begin{itemize}
    \item If a header is included several times, this prevents
      multiply definitions of types
      \begin{lstlisting}
#ifndef __MYHEADER_HPP
#define __MYHEADER_HPP
.
// The actual code
.
#endif          
      \end{lstlisting}
    \item Or they can be used for conditional compilation, \eg when a
      program can use a library if it is there, but can still work
      if not
      \begin{lstlisting}
#ifdef FFTW
// code that uses FFTW
#else
// code without FFTW
#endif        
      \end{lstlisting}
    \end{itemize}
\end{frame}

\begin{frame}[fragile]
  \frametitle{Stage 2: Compiler}

  \begin{itemize}
  \item Performs \emph{static type checking}: do the types match?
  \item The \emph{code generator} translates source code into machine
    code, function by function
  \item An \emph{optimizer} optimizes the machine code
  \item Complains when a symbol is used that has not been declared
  \item \dots but does \emph{not} complain when it is not defined!
  \item It generates \lstinline!.o!-files (\emph{object
      code}-files)\footnote{These objects have nothing to do with
      objects from OOP!}
    \begin{itemize}
    \item For each \emph{defined symbol} in the code, it will store
      the generated machine code together with the symbol
    \item For each \emph{undefined symbol} that is used in the code
      and that was only declared but not defined , it will store where
      it is used
    \end{itemize}
  \item \lstinline!g++ -c FILE!: Preprocess and compile FILE, output
    \lstinline!.o!-file
  \item \lstinline!nm -C FILE.o!: Shows defined and undefined symbols
    in \lstinline!.o!-file
  \end{itemize}
\end{frame}

\begin{frame}[fragile]
  \frametitle{Stage 3: Linker}

  \begin{itemize}
  \item \emph{Links} several \lstinline!.o!-files together into a
    single executable file
  \item Starts with the function \texttt{main}
  \item Recursively \emph{resolves} all undefined symbols
    \begin{itemize}
    \item puts the code of used symbols into the executable
    \item puts the addresses of the symbol where the symbol is used
    \end{itemize}
  \item Fails when a symbol cannot be resolved
  \end{itemize}
\end{frame}

\begin{frame}[fragile]
  \frametitle{Libraries}
  \begin{itemize}
  \item \lstinline!ar! can put several object files together into a
    \emph{library}
  \item The name is \lstinline!lib<name>.a! or
    \lstinline!lib<name>.so! (\eg \lstinline!libgsl.a!)
  \item A library file should come together with a set of header files
    (\eg gsl.h)
  \item A library file can be linked together with other object files
    via the compiler option \texttt{-l<name>} (\eg \texttt{-lgsl})
  \end{itemize}
\end{frame}


\end{document}
