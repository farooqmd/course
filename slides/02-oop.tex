\documentclass{slides}

\usepackage{tikz-uml}

\lstset{language=C++}

\begin{document}

\graphicspath{{figures/}}

\title[Object-oriented Programming]{\Large Object-oriented Programming}

\author[A. Arnold and O. Lenz]{Axel Arnold \and Olaf Lenz} 
\institute{Institut für Computerphysik\\Universit\"at Stuttgart}
\date{February 18-22, 2013}

\setbeamertemplate{footline}{}
\begin{frame}
  \titlepage
\end {frame}
\setbeamertemplate{footline}[icp]

\begin{frame}
  \frametitle{Outline}

  \begin{itemize}
  \item Basic ideas
  \item UML -- Universal Markup Language
  \item Classes, Interfaces and Methods
  \item Objects/Instances
  \item Composition/Fields
  \item Inheritance
  \item Polymorphism
  \end{itemize}

  \begin{block}{Quellen}
    \begin{itemize}
    \item Bruce Eckels, ``Thinking in C++'', 2nd edition, Prentice-Hall
    \item Martin Fowler, ``UML Distilled'', 3rd edition, Addison-Wesley
    \end{itemize}
  \end{block}
  
\end{frame}

\begin{frame}
  \frametitle{Object-oriented Programming (OOP)}

  \begin{itemize}
  \item OO is a language-indedependent concept
  \item in principle not limited to programming (OO design)\\
    $\implies$ databases, business plans
  \item improves reusability and exchangability of code
  \item separation of partial problems
  \item ``real world'' modelling
  \item representation in \alert{Universal Markup Language} (UML)
  \end{itemize}
  \medskip

  \tikzumlset{font=\tiny}
  \begin{tikzpicture}[scale=1.1]
    \begin{umlsystem}[x=4, fill=red!10]{C++ course}
      \umlusecase{Lectures}
      \umlusecase[x=4]{Chapters}

      \umlusecase[y=-1]{Exercises}
      \umlusecase[x=4, y=-1]{Small tasks}
      \umlusecase[x=3, y=-1.5]{Project}
    \end{umlsystem}

    \umlactor[scale=0.5]{Olaf/Axel}
    \umlactor[scale=0.5,y=-1.5]{Participant}

    \umlassoc{Participant}{usecase-1}
    \umlassoc{Olaf/Axel}{usecase-1}
    \umlassoc{Participant}{usecase-3}
    \umlassoc{Olaf/Axel}{usecase-3}
    \umlinclude{usecase-1}{usecase-2}
    \umlinclude{usecase-3}{usecase-4}
    \umlinclude{usecase-3}{usecase-5}
  \end{tikzpicture}
\end{frame}

\begin{frame}
  \frametitle{Basic ideas}

  \begin{itemize}
  \item \alert{Everything is an object}\\
    \emph{game board, Axel, Olaf, their playing pieces, monster}
  \item \alert{Objects interact by sending/receiving messages}\\
    \emph{Axel $\rightarrow$ playing piece: move one east}\\
    \emph{Olaf $\rightarrow$ game board: what is the object at
      position A,B?}\\
  \item \alert{An object consists of objects}\\
    \emph{the game board consists of $12\times 12$ tiles}\\
    \emph{Axel has 3 playing pieces in control}\\
  \item \alert{Every object has a type}\\
    \emph{Axel is a player}\\
    \emph{the game board is a ``Fearsome floors'' game board}
  \item \alert{All objects of the same type understand the same messages}\\
    \emph{all playing pieces can move on the board}\\
    \emph{Axel and Olaf can move their playing pieces}
  \end{itemize}
\end{frame}

\begin{frame}
  \frametitle{Basic ideas -- in C++}

  \begin{itemize}
  \item \alert{Everything is an object}\\
    \emph{objects are variables}
  \item \alert{Objects interact by sending/receiving messages}\\
    \emph{calling methods (functions) that act on an object (variable)}\\
    \emph{exchanged information in arguments and return values}
  \item \alert{An object consists of objects}\\
    \emph{structured objects/records}\\
    \emph{smallest units: C datatypes, int, char, ...}\\
  \item \alert{Every object has a type}\\
    \emph{classes or plain old data types in C++}\\
    \emph{types can be derived from others }
  \item \alert{All objects of the same type understand the same messages}\\
    \emph{class declaration lists possible messages}\\
    \emph{messages can be inherited}
  \end{itemize}
\end{frame}

\begin{frame}
  \only<1>{\frametitle{Classes, Interfaces and Methods}}
  \only<2>{\frametitle{Objects/Instances}}

  \begin{center}
    \begin{tikzpicture}
      \umlclass[type=abstract]{Player}{ }{
        getName() : string\\[\medskipamount]
        setPlayerNum(playerNum : num)\\
        getPlayerNum() : int\\[\medskipamount]
        doNextMove(activePiece : Piece, event : Event, out message : string)\\
      }
    \end{tikzpicture}
  \end{center}

  \begin{onlyenv}<1>
    \begin{itemize}
    \item classes describe the type of objects
    \item classes define the \alert{interface} of objects
    \item the interface consists of \alert{methods}
    \item methods are just normal functions that take\\
      arguments and return values of arbitrary types
    \end{itemize}
  \end{onlyenv}

  \begin{onlyenv}<2>
    \begin{itemize}
    \item an object is created by naming it and its class
    \item compare: declaration of a variable in C/C++
    \end{itemize}
  \end{onlyenv}
\end{frame}

\begin{frame}
  \frametitle{Overloading and signature}
  
  \begin{center}
    \begin{tikzpicture}
      \umlclass[type=abstract]{Tile}{ }{
        \umlvirt{getRepresentation(representation : string)}\\[\medskipamount]
        \umlvirt{tryStep(piece : Piece, how : Event, out message : string)}\\
        \umlvirt{tryStep(monster : Monster, how : Event, out message : string)}\\
      }
    \end{tikzpicture}
  \end{center}

  \begin{itemize}
  \item a method is described by name and \alert{signature}
  \item signature is formed by the types of all taken arguments\\
    \emph{piece : Piece, how : Event, out message : string}\\
    \emph{monster : Monster, how : Event, out message : string}
  \item return value is \emph{not} part of the signature
  \end{itemize}
\end{frame}

\begin{frame}
    \frametitle{Composing classes}

  \begin{center}
    \begin{tikzpicture}
      \umlclass[type=abstract]{Player}{
        playerNum : int\\
        name : string
      }{
        setPlayerNum(int)\\
        getPlayerNum() : int
      }

      \umlclass[x=6]{Piece}{
        x, y: int
      }{
        getPosition(): int, int
      }
      \umlunicompo[mult2=*,anchors=10 and 170]{Player}{Piece}
      \umluniassoc[mult2=1,anchors=-170 and -10]{Piece}{Player}
    \end{tikzpicture}
  \end{center}

  \begin{onlyenv}<1>
    \begin{itemize}
    \item objects are made of objects (\alert{attributes}) ---\\
      classes declare the types of these objects
    \item ``has-a'' or ``has-many'' relations
    \item simple attributes appear below class name
    \item complex classes shown as \alert{composition}\\
      \emph{a player has many playing pieces},
      \emph{a piece belongs to one player}
    \end{itemize}
  \end{onlyenv}

\end{frame}

\begin{frame}
  \only<1>{\frametitle{Inheritance and class hierarchy}}
  \only<2>{\frametitle{Polymorphism}}
  \only<3>{\frametitle{Abstract classes}}
  \only<4>{\frametitle{Usability}}

  \begin{center}
    \begin{tikzpicture}
      \umlclass[y=2,x=4,type=abstract]{Tile}{
        game: Game
      }{
        \umlvirt{tryStep(...)}
      }
      \umlclass{FloorTile}{
      }{
        tryStep(...)
      }

      \umlclass[x=8]{WallTile}{
      }{
        tryStep(...)
      }

      \umlinherit{WallTile}{Tile}
      \umlinherit{FloorTile}{Tile}
    \end{tikzpicture}
  \end{center}
  \begin{onlyenv}<1>
    \begin{itemize}
    \item classes often are specializations of other classes\\
      \emph{a human is a mammal is an animal ...}
    \item \alert{hierarchy of classes}
    \item reusability: implement the common properties only once
    \end{itemize}
  \end{onlyenv}
  \begin{onlyenv}<2>
    \begin{itemize}
    \item subclasses support the same interface
    \item subclasses inherit methods and attributes of all superclasses
    \item .. but methods might behave differently
      (\alert{polymorphism})
    \item in C++: explicitely declared by keyword \alert{virtual}
    \end{itemize}
  \end{onlyenv}
  \begin{onlyenv}<3>
    \begin{itemize}
    \item problem: \texttt{Tile} cannot implement \texttt{step}\\
      since  all game tiles behave differently when stepped on
    \item answer: \alert{abstract class}es implement only parts of the interface
    \item in C++: a class with undeclared virtual functions is abstract
    \end{itemize}
  \end{onlyenv}
  \begin{onlyenv}<4>
    \begin{alertbox}{0.75\textwidth}
      any object of a derived class can be used like an object of any
      base class
    \end{alertbox}
  \end{onlyenv}
\end{frame}

\begin{frame}
  \frametitle{Implementation hiding}
  \tikzumlset{font=\tiny}
  \begin{center}
    \begin{tikzpicture}
      \umlclass[type=abstract]{Tile}{
        \# game: Game
      }{
        \umlvirt{+ tryStep(...)}
      }
      \umlclass[y=-2]{WallTile}{
      }{
        + tryStep(...)
      }
      \umlclass[x=4,y=-1]{Player}{
        - playerNum: int
      }{
      }
      
      \umlinherit{WallTile}{Tile}
      \umldep{Player}{WallTile}
      \umldep{Player}{Tile}
    \end{tikzpicture}
  \end{center}

  \begin{itemize}
  \item \alert{private} attributes and methods are \emph{not} visible to
    other objects
  \item \alert{public} elements are visible to all
  \item \alert{protected} elements are only visible to derived classes
  \item \texttt{game} is \alert{protected} $\implies$ only visible to
    \texttt{WallTile}, not to \texttt{Player}
  \item \texttt{Player} can access \texttt{tryStep} of all tiles
  \item \texttt{playerNum} is not visible to anybody
  \end{itemize}
\end{frame}

\end{document}
