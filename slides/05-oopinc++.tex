\documentclass{slides}
\usepackage{tikz-uml}

\lstset{language=C++}

\begin{document}

\graphicspath{{figures/}}

\title[Object-oriented Programming in C++]{\Large Object-oriented
  Programming in C++}

\author[A. Arnold and O. Lenz]{Axel Arnold \and Olaf Lenz} 
\institute{Institut für Computerphysik\\Universit\"at Stuttgart}
\date{February 18-22, 2013}

\setbeamertemplate{footline}{}
\begin{frame}
  \titlepage
\end {frame}
\setbeamertemplate{footline}[icp]

\begin{frame}
  \frametitle{Outline}
  \begin{itemize}
  \item Definition of classes
  \item Lifetime of objects
  \item Constructor, destructor
  \item Default and copy constructor
  \item Encapsulation: private, public
  \item Friends
  \item Function and operator overloading
  \item Inheritance
  \item Protected encapsulation
  \item Polymorphism and virtual functions
  \item Abstract base classes and pure virtual functions
  \item Multiple inheritance
  \end{itemize}
\end{frame}

\begin{frame}[fragile]
  \frametitle{Definition of classes}

  \begin{onlyenv}<1> \tikzumlset{font=\tiny}
    \begin{tikzpicture}
      \umlclass{Player}{
        playerNum : int\\
        name : string }{
        setPlayerNum(playerNum : num)\\
        getPlayerNum() : int\\[\medskipamount]
        doNextMove(piece : Piece, event : Event, out message :
        string) }
    \end{tikzpicture}
    ~~becomes
    \medskip
\end{onlyenv}

\begin{lstlisting}[emph={class,public}]
class Player {
public:
  int playerNum;
  string name;
  // setter and getter for the player number
  void setPlayerNum(int playerNum);
  int getPlayerNum();
  // perform the next move
  void doNextMove(Piece &piece, Event event, string &msg);
};
\end{lstlisting}

\begin{onlyenv}<2>
  \begin{itemize}
  \item \alert{remember semicolon at the end of the definition!}
  \item \lstinline!public! will be explained later
  \item output parameters translate to references (\lstinline!message!)
  \item no separator between member functions and variables required
  \item good style to separate them
  \end{itemize}
\end{onlyenv}
\end{frame}

\begin{frame}[fragile]
  \frametitle{(Forward) declaration}

This compiles as is:
\begin{lstlisting}
#include <vector>
#include <string>
using namespace std;
enum Event { North, East, South, West, Skip };
class Piece; // <-

class Player {
public:
  vector<Piece *> pieces;
  void doNextMove(Piece &piece, Event event, string &msg);
};
\end{lstlisting}

  \begin{itemize}
  \item classes can be (forward) \alert{declared} (here: \lstinline!Piece!)
  \item necessary if \lstinline!Piece! uses \lstinline!Player!
    and vice versa
  \item ok as long as only addresses are needed (pointers or references)
  \item \lstinline!enum Event! however has to be defined
  \end{itemize}
  
\end{frame}


\begin{frame}[fragile]
  \frametitle{A more complex example}

  \begin{tikzpicture}
    \umlclass[type=abstract]{Player}{
      playerNum : int\\
      name : string
    }{}

    \umlclass[x=6]{Piece}{
      x, y: int
    }{}
    \umlunicompo[arg2=*,anchors=10 and 170]{Player}{Piece}
    \umluniassoc[arg2=1,anchors=-170 and -10]{Piece}{Player}
  \end{tikzpicture}
  ~~becomes
  \medskip

\begin{lstlisting}
class Player; // <-
class Piece {
  int x, y;
  Player &player;
};
class Player {
  int playerNum;
  vector< Piece* > pieces;
};
\end{lstlisting}

  \begin{itemize}
  \item here, a forward declaration is obviously necessary
  \item Usually, declare the higher level class forward
  \end{itemize}
\end{frame}

\begin{frame}[fragile]
  \frametitle{Definition of member functions}

  \begin{onlyenv}<1-2>
\begin{lstlisting}[title=Player.hpp]
class Player {
  void doNextMove(Piece &piece, Event ev, string &msg);
  int getPlayerNum() { return playerNum; }
};
\end{lstlisting}

\begin{lstlisting}[title=Player.cpp]
void
Player::doNextMove(Piece &piece, Event ev, string &msg) {
  piece.tryStep(ev, msg);
}
\end{lstlisting}
  \end{onlyenv}
  \begin{onlyenv}<3-4>
\begin{lstlisting}[title=Player.hpp]
class Player {
  void doNextMove(Piece *piece, Event ev, string &msg);
  int getPlayerNum() { return playerNum; }
};
\end{lstlisting}

\begin{lstlisting}[title=Player.cpp]
void
Player::doNextMove(Piece *piece, Event ev, string &msg) {
  piece->tryStep(ev, msg);
}
\end{lstlisting}
  \end{onlyenv}

  \begin{onlyenv}<1>
    \begin{itemize}
    \item member functions are defined like normal functions
    \item a member function is referred by its name and class
    \item class name and member name are separated by ``::''
    \end{itemize}
  \end{onlyenv}

  \begin{onlyenv}<2>
    \begin{itemize}
    \item member functions are called by ``object.function()''
    \item object name and member name are separated by ``.''
    \item same holds for member variables
    \end{itemize}
  \end{onlyenv}

  \begin{onlyenv}<3>
    \begin{itemize}
    \item with a pointer to an object, you can use ``->'' like in C
    \item ``(*pointer).function()'' is the same as ``pointer->function()''
    \item almost true due to operator overloading (later)
    \end{itemize}
  \end{onlyenv}

  \begin{onlyenv}<4>
    \begin{itemize}
    \item class definitions are usually placed in header file ``\textit{class}.hpp''
    \item member function definitions in separate cpp-file ``\textit{class}.cpp''
    \item \emph{small} functions can be defined in the class definition (\alert{inlined})
    \end{itemize}
  \end{onlyenv}

\end{frame}

\begin{frame}[fragile]
  \frametitle{Referring to yourself: this}

\begin{lstlisting}[emph={this}]
void Piece::register() {
  board.addPiece(this);
}
void Piece::escape() {
  board.removePiece(this);
}
\end{lstlisting}

  \begin{block}{}
    How can an object refer to ``itself''?\\
    \emph{\lstinline!Piece! registers itself with the board}
  \end{block}

  \begin{block}{Answer}
    In a member function, \lstinline!this! always points to the
    current object
  \end{block}

  \begin{itemize}
  \item \lstinline!this! is a pointer, although guaranteed to be valid
  \item there is no reference to the current object, use \lstinline!*this!
    if necessary
  \end{itemize}
\end{frame}

\begin{frame}[fragile,fragile]
  \frametitle{Creation and life time of objects}

  \begin{onlyenv}<1>
\begin{lstlisting}
int test() {
  for (int i = 0; i < 10; ++i) {
    Piece piece;
    piece.register();
    // here, piece is destroyed automatically
  }
}
\end{lstlisting}

    \begin{itemize}
    \item objects are created like variables in C by giving type and name
    \item ...and die at the end of the scope (code block)
    \item exception: static variables like in C
    \item therefore, after the loop above, not a single piece exists
    \end{itemize}
  \end{onlyenv}

  \begin{onlyenv}<2>
\begin{lstlisting}
int test() {
  for (int i = 0; i < 10; ++i) {
    Piece *piecePtr = new Piece;
    piecePtr->register();
    delete piecePtr;
    // without the older pieces remain, inaccessible
  }
}
\end{lstlisting}

    \begin{itemize}
    \item objects can also be created by \lstinline!new!
    \item these live till they are destroyed by explicitely calling
      \lstinline!delete! \emph{once}
    \item there should be \emph{one} responsible object for
      destroying, the \alert{owner}
    \item more than one owner leads to segmentation faults
    \item no owner (and therefore no \lstinline!delete!) to memory leaks
    \item in OOP, ownership frequently changes, making things
      difficult
    \end{itemize}
  \end{onlyenv}
\end{frame}

\begin{frame}[fragile]
  \frametitle{Constructor}

\begin{lstlisting}
class Game {
  Board* board;
  vector< Player* > players;
public:
  Game(int numPlayers = 2): board(0) {
    for (int i = 0; i < numPlayers; ++i)
      players.push_back(new Player);
  }
};
\end{lstlisting}

  \begin{itemize}
  \item constructor is a function initializing an object
  \item has the same name as the classs (``Class::Class'')
  \item can take arguments that specify what to create
  \item there can be several constructors (see overloading)
  \item initialize member variables after colon (here: \lstinline!board!)
  \item multiple variables separated by comma
  \end{itemize}
\end{frame}

\begin{frame}[fragile]
  \frametitle{Destructor}

\begin{lstlisting}
class Game {
  Board* board;
  vector< Player* > players;
public:
  ~Game() {
    for (auto player: players) delete player;
    delete board;
  }
};
\end{lstlisting}

  \begin{itemize}
  \item destructor is called right \emph{before} unallocating the
    memory
  \item should clean up
  \item in particular destroy all owned objects
  \item inform other objects that have pointers to this object
  \item the destructor cannot take arguments
  \end{itemize}
\end{frame}

\begin{frame}[fragile]
  \frametitle{Default and copy constructor}

\begin{lstlisting}
class Game {
public:
  Game()      { /* call default constructors of all objects */ }
  Game(const Game &src)  { /* use copy constructors instead */ }
};
\end{lstlisting}

  \begin{itemize}
  \item two \emph{predefined} constructors:
    \begin{itemize}
    \item Default constructor \lstinline!Class::Class()!
      \begin{itemize}
      \item only if no user-defined constructor
      \item initializes all objects using their default constructor
      \item plain old data types (char, int, ...) are potentially
        uninitialized
      \item raises a warning in this case
      \end{itemize}
    \item Copy constructor \lstinline!Class::Class(const Class &src)!
      \begin{itemize}
      \item copies all member variables from object \lstinline!src!
      \item for pointers often a bad idea - ownership is not clear
      \item also for \emph{big} classes
      \end{itemize}
    \end{itemize}
  \item to avoid automatic generation, declare the functions
    \emph{private}
  \item you don't need to specify them
  \end{itemize}
\end{frame}

\begin{frame}[fragile]
  \frametitle{Static class members}

\begin{lstlisting}[emph={static,getCount}]
class CountOccurance {
  static int cnt;
public:
  CountOccurance() { ++cnt; }
  ~CountOccurance() { --cnt; }
  static int getCount() { return cnt; }
};
int CountOccurance::cnt = 0;
int main() {
  new CountOccurance();
  cout << CountOccurance::getCount() << endl;
  return 0;
}
\end{lstlisting}

  \begin{itemize}
  \item class members (functions and variables) can be static
  \item variable needs to be defined once somewhere (not in header!)
  \item static member functions are called with \emph{class} prefix
  \end{itemize}

\end{frame}

\begin{frame}[fragile]
  \frametitle{Function overloading}
  
  \begin{center}
    \begin{tikzpicture}
      \umlclass[type=abstract]{Tile}{ }{
        \umlvirt{tryStep(piece : Piece, how : Event, out message : string)}\\
        \umlvirt{tryStep(monster : Monster, how : Event, out message : string)}\\
      }
    \end{tikzpicture}
  \end{center}
  becomes
\begin{lstlisting}
class Tile {
public:
  void tryStep(Piece &piece, Event how, string &message);
  void tryStep(Monster &monster, Event how, string &message);
};
\end{lstlisting}

  \begin{itemize}
  \item several methods with same name but different signature
  \item signature is formed by the types of all taken arguments
  \item return value is \emph{not} part of the signature
  \item also global (C-style) functions can be overloaded
  \end{itemize}
\end{frame}

\begin{frame}[fragile]
  \frametitle{Default arguments}
  
\begin{lstlisting}
class QGameBoard
{
public:
  QGameBoard(int tileSize = 20, QWidget *parent = 0);
  void setTileSize(int sizeX, int sizeY = 0);
};
\end{lstlisting}

  \begin{itemize}
  \item another kind of function overloading are default arguments
  \item works for any kind of functions, including constructors
  \item values need not to be specified when calling
  \item always starts from the back:\\
    \lstinline!QGameBoard(30)! is ok: \lstinline!parent = 0!\\
    \lstinline!QGameBoard(otherWidget)! not:
    \lstinline!parent = 0!, the pointer \lstinline!otherWidget! is cast into an
    integer for \lstinline!tileSize!
  \item here, this also overloads the default constructor
    \lstinline!QGameBoard()!, which is therefore not generated
  \end{itemize}

\end{frame}

\begin{frame}[fragile]
  \frametitle{Operator overloading}

  \begin{itemize}
  \item most operators can be overloaded
  \item abuse can lead to \emph{big} confusion
  \item obviously relies heavily on signatures
  \end{itemize}

Some examples:
\begin{lstlisting}
// output using iostream
ostream &operator<<(ostream &stream, const Class &a);
// assignment of any type:
Class &Class::operator=(OtherClass b);
// sum (product, difference,... analogously):
RetClass Class::operator+(OtherClass b);
// ++object:
Class &Class::operator++();
// object++ (yes, they can differ):
Class Class::operator++(int); // int is a dummy argument!
// for objects representing functions:
RetClass Class::operator()(ParamClass a, ParamClass b, ...);
\end{lstlisting}

\end{frame}

\begin{frame}[fragile]
  \frametitle{Inheritance}
  \tikzumlset{font=\tiny}
  \begin{tikzpicture}
    \umlclass{Tile}{
      x, y : int
    }{
      getPosition(out int, out int)\\
      tryStep(...)
    }
      
    \umlclass[x=5]{WallTile}{
    }{
      tryStep(...)
    }

    \umlinherit{WallTile}{Tile}
  \end{tikzpicture}
  ~~becomes
  \medskip
\begin{lstlisting}[emph={public}]
class Tile {
  int x, y;
  void getPosition(int x, int y);
  virtual void tryStep(Piece &piece, Event how, string &message);
};
class WallTile : public Tile {
  virtual void tryStep(Piece &piece, Event how, string &message);
};
\end{lstlisting}

  \begin{itemize}
  \item derived classes declare super classes after class name and colon
  \item getPosition also exists in WallTile (\alert{inherited} from \lstinline!Tile!)
  \end{itemize}
\end{frame}

\begin{frame}[fragile]
  \frametitle{Constructor and base classes}

\begin{lstlisting}
class Tile {
public:
  Tile();
  ~Tile();
};
class WallTile: public Tile {
public:
  WallTile(): Tile() { }  
  WallTile(int) { }  
  ~WallTile();
};
\end{lstlisting}

  \begin{onlyenv}<1>
    \begin{itemize}
    \item base classes are initialized in constructor following colon\\
      (like member variables)
    \item using default constructors can be omitted (\lstinline!WallTile(int)!)
    \item arguments to constructors can be computed freely
    \item base classes are automatically destroyed
    \end{itemize}
  \end{onlyenv}

  \begin{onlyenv}<2>
    \begin{itemize}
    \item order follows matryoshka principle: from general to special
    \item base class constructors come \emph{first}
    \item base class destructors come \emph{last}
    \end{itemize}
  \end{onlyenv}
\end{frame}

\begin{frame}[fragile]
  \frametitle{Encapsulation: private, public, protected}

  \begin{onlyenv}<1>
    \tikzumlset{font=\tiny}
    \begin{center}
      \begin{tikzpicture}
        \umlclass[type=abstract]{Tile}{
          - notvisible: bool\\
          \# x, y: int }{ + getPosition(out int, out int) }
        \umlemptyclass[y=-1,x=3.5]{WallTile}
        \umlemptyclass[x=6]{Player}
      
        \umlinherit{WallTile}{Tile} \umldep{Player}{WallTile}
        \umldep{Player}{Tile}
      \end{tikzpicture}
    \end{center}
    \vspace{-\baselineskip}
    becomes
    \medskip
  \end{onlyenv}

\begin{lstlisting}[emph={public,private,protected}]
class Tile {
private:
  bool notvisible;
protected:
  int x, y;
public:
  void getPosition(int x, int y);
};
class WallTile : public Tile { };
\end{lstlisting}

  \begin{onlyenv}<2>
    \begin{itemize}
    \item \alert{private} members are only accessible to members of
      the class
    \item \alert{protected} is also accessible to members of derived
      classes
    \item \alert{public} members are visible to all
    \item no difference between function and variable members
    \item a non-public destructor hinders destroying objects ---
      rarely a good idea
    \end{itemize}
  \end{onlyenv}

  \begin{onlyenv}<3>
    \begin{itemize}
    \item \lstinline!notvisible! is only accessible to
      \lstinline!Tile!  instances (\alert{private})
    \item \lstinline!x! and !y! are accessible to instances of classes
      \lstinline!Tile! and \lstinline!WallTile! (\alert{protected})
    \item \lstinline!getPosition! is visible to all, including
      e.g. \lstinline!Player! instances (\alert{public})
    \end{itemize}
  \end{onlyenv}

  \begin{onlyenv}<4>
    \begin{itemize}
    \item base classes also have a visibility (private, protected or public)
    \item most stringent visibility applies
    \item private members never become protected or public
    \item protected members never become public
    \end{itemize}
  \end{onlyenv}

  \begin{onlyenv}<5>
    \begin{itemize}
    \item objects are always also base class objects\\
      \emph{(a \lstinline!WallTile! is a \lstinline!Tile!, a
      \lstinline!Mammal! an \lstinline!Animal!)}\\
      $\implies$ \alert{base class (interface) should be public}
    \item ... if not strong reasons against
    \item Qt: $\approx 2300$ public inheritances, $\approx 30$ private
      and $\approx 10$ protected
    \end{itemize}
  \end{onlyenv}
\end{frame}

\begin{frame}[fragile]
  \frametitle{Friends}

\begin{lstlisting}[emph={friend}]
class Hidden {
private:
  Hidden();
  Hidden(const Hidden &);
  ~Hidden();
  friend ostream &operator<<(ostream &, const Hidden &);
  friend class HiddenFactory;
};
\end{lstlisting}
  \begin{itemize}
  \item \lstinline!friend! overrides any visibility restrictions
  \item friends are declared in the class definition\\
    $\implies$ you cannot inject friends
  \item friends can be classes or (member) functions
  \item common use is allowing the \lstinline!<<!-operator to access inner parts
  \item here, it enforces to use \lstinline!HiddenFactor! to create objects
  \end{itemize}
\end{frame}

\begin{frame}[fragile]
  \frametitle{Polymorphism: virtual functions}

  \begin{onlyenv}<1>
\begin{lstlisting}
class Tile {
  string whatAmI() { return "tile"; }
};
class FloorTile: public Tile {
  string whatAmI() { return "floor tile"; }
};
void informMe(Tile &tile) {
  cout << "this is a " << tile.whatAmI() << endl;
}
\end{lstlisting}

    \begin{itemize}
    \item doesn't work --- \lstinline!informMe! always says it is a ``tile''
    \item even for a \lstinline!FloorTile!, \lstinline!Tile::whatAmI!
      is called, since \lstinline!tile! is a (reference to) \lstinline!Tile!
    \item when accessed through a base class pointer or reference,
      the object ``forgets'' its true type
    \end{itemize}
  \end{onlyenv}

  \begin{onlyenv}<2>
\begin{lstlisting}[emph={virtual}]
class Tile {
  virtual string whatAmI() { return "tile"; }
};
class FloorTile: public Tile {
  virtual string whatAmI() { return "floor tile"; }
};
void informMe(Tile &tile) {
  cout << "this is a " << tile.whatAmI() << endl;
}
\end{lstlisting}

    \begin{itemize}
    \item \lstinline!virtual! declares a function to be \alert{polymorph}
    \item it acts differently depending on the \emph{actual} type of the object
    \item therefore, \lstinline!Tile::whatAmI! is different for a
      \lstinline!Tile! and a \lstinline!WallTile!
    \item classes with virtual functions carry a pointer to a table of
      virtual functions (vtable)
    \item forgetting to define a virtual function leads to undefined
      references to the ``vtable for Class''
    \end{itemize}
  \end{onlyenv}
\end{frame}

\begin{frame}[fragile]
  \frametitle{Pure virtual functions and abstract base classes}

\begin{lstlisting}
class Tile {
  virtual string whatAmI() = 0;
};
class FloorTile: public Tile {
  virtual string whatAmI() { return "floor tile"; }
};
int main() {
  // Error! Abstract class, cannot be instantiated
  Tile tile;
}
\end{lstlisting}

    \begin{itemize}
    \item pure virtual (abstract) functions are not be defined for the
      base class, only for derived classes
    \item classes with pure virtual functions are \alert{abstract}
      classes
    \item you cannot create objects of abstract classes
    \end{itemize}
\end{frame}

\begin{frame}[fragile]
  \frametitle{Multiple inheritance}
  
  \begin{onlyenv}<1>
    \tikzumlset{font=\tiny}
    \begin{tikzpicture}
      \umlclass{Person}{ name: string }{} \umlclass[x=5]{Horse}{
        tailLength : float }{} \umlemptyclass[x=2.5,y=-1]{Centaur}

      \umlinherit{Centaur}{Person} \umlinherit{Centaur}{Horse}
    \end{tikzpicture}
    ~~becomes
  \end{onlyenv}

\begin{lstlisting}[emph={public}]
class Person {
  string name;
  Person(const string &n): name(n) {}
};
class Horse {
  float tailLength;
  Horse(float t): tailLength(t) {}
};
class Centaur: public Person, public Horse {
  Centaur(const string &n, float t) : Person(n), Horse(t) {}
};
\end{lstlisting}

  \begin{onlyenv}<2>
    \begin{itemize}
    \item multiple base classes are separated by commas
    \item each carries a visibility (public, protected or private)
    \item class has member variables and functions of both base
      classes
    \end{itemize}
  \end{onlyenv}
\end{frame}

\begin{frame}
  \frametitle{The diamond problem}

  \begin{center}
    \tikzumlset{font=\tiny}
    \begin{tikzpicture}
      \umlclass[x=3]{Person}{
        name: string
      }{
        getNameAndTitle() : string
      }
      \umlclass[y=-2.5]{Student}{}{
        \umlvirt{getNameAndTitle() : string}
      }
      \umlclass[y=-2.5,x=6]{Teacher}{}{
        \umlvirt{getNameAndTitle() : string}
      }
      \umlclass[y=-5,x=3]{StudyingTeacher}{}{
        \umlvirt{getNameAndTitle() : string}
      }
      
      \umlinherit{StudyingTeacher}{Teacher}
      \umlinherit{StudyingTeacher}{Student}
      \umlinherit{Teacher}{Person}
      \umlinherit{Student}{Person}
    \end{tikzpicture}
  \end{center}
\end{frame}

\begin{frame}[fragile]
  \frametitle{The diamond problem --- solution in C++}

\begin{lstlisting}[basicstyle={\ttfamily\tiny},emph={virtual}]
class Person {
protected: string name;
public:    Person(const string &_name): name(_name) { }
           virtual string getNameAndTitle();
};
class Teacher: virtual public Person {
public:    Teacher(const string &_name): Person(_name) { }
           virtual string getNameAndTitle();
};
class Student: virtual public Person {
public:    Student(const string &_name): Person(_name) { }
           virtual string getNameAndTitle();
};
class StudyingTeacher
  : virtual public Teacher, virtual public Student {
public:    StudyingTeacher(const string &n)
             : Person(n), Teacher(n), Student(n) {}
           virtual string getNameAndTitle();
};
\end{lstlisting}

  \begin{itemize}
  \item declare base classes virtual
  \item requires also to initialize the common ancestor in the constructor
  \item initializiation of ancestor in intermediate classes is skipped
  \end{itemize}
\end{frame}

\end{document}
