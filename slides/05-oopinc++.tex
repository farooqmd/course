\documentclass{slides}
\usepackage{tikz-uml}

\lstset{language=C++}

\begin{document}

\graphicspath{{figures/}}

\title[Object-oriented Programming in C++]{\Large Object-oriented
  Programming in C++}

\author[A. Arnold and O. Lenz]{Axel Arnold \and Olaf Lenz} 
\institute{Institut für Computerphysik\\Universit\"at Stuttgart}
\date{February 18-22, 2013}

\setbeamertemplate{footline}{}
\begin{frame}
  \titlepage
\end {frame}
\setbeamertemplate{footline}[icp]

\begin{frame}
  \frametitle{Outline}
  \begin{itemize}
  \item Definition of classes
  \item Lifetime of objects
  \item Constructor, destructor
  \item Default and copy constructor
  \item Encapsulation: private, public
  \item Friends
  \item Function and operator overloading
  \item Inheritance
  \item Protected encapsulation
  \item Polymorphism and virtual functions
  \item Abstract base classes and pure virtual functions
  \item Multiple inheritance
  \end{itemize}
\end{frame}

\begin{frame}[fragile]
  \frametitle{Definition of classes}

  \begin{onlyenv}<1> \tikzumlset{font=\tiny}
    \begin{tikzpicture}
      \umlclass{Player}{
        playerNum : int\\
        name : string }{
        setPlayerNum(playerNum : num)\\
        getPlayerNum() : int\\[\medskipamount]
        doNextMove(piece : Piece, Event : event, out string :
        message) }
    \end{tikzpicture}
    ~~becomes
    \medskip
\end{onlyenv}

\begin{lstlisting}
class Player {
public:
  vector<Piece *> pieces;
  int playerNum;
  // setter and getter for the player number
  void setPlayerNum(int playerNum);
  int getPlayerNum();
  // perform the next move
  void doNextMove(Piece &piece, Event event, string &msg);
};
\end{lstlisting}

\begin{onlyenv}<2>
  \begin{itemize}
  \item \alert{remember semicolon at the end of the definition!}
  \item \lstinline!public! will be explained later
  \item output parameters translate to references (\lstinline!message!)
  \item no separator between member functions and variables required
  \item good style to separate them
  \end{itemize}
\end{onlyenv}
\end{frame}

\begin{frame}[fragile]
  \frametitle{(Forward) declaration}

This compiles as is:
\begin{lstlisting}
#include <vector>
#include <string>
using namespace std;
enum Event { North, East, South, West, Skip };
class Piece;

class Player {
public:
  vector<Piece *> pieces;
  void doNextMove(Piece &piece, Event event, string &msg);
};
\end{lstlisting}

  \begin{itemize}
  \item classes can be (forward) \alert{declared} (here: \lstinline!Piece!)
  \item necessary if \lstinline!Piece! uses \lstinline!Player!
    and vice versa
  \item ok as long as only addresses are needed (pointers or references)
  \item \lstinline!enum Event! however has to be defined
  \end{itemize}
  
\end{frame}


\begin{frame}[fragile]
  \frametitle{A more complex example}

  \begin{tikzpicture}
    \umlclass[type=abstract]{Player}{
      playerNum : int\\
      name : string
    }{}

    \umlclass[x=6]{Piece}{
      x, y: int
    }{}
    \umlunicompo[arg2=*,anchors=10 and 170]{Player}{Piece}
    \umluniassoc[arg2=1,anchors=-170 and -10]{Piece}{Player}
  \end{tikzpicture}
  ~~becomes
  \medskip

\begin{lstlisting}
class Player;
class Piece {
  int x, y;
  Player &player;
};
class Player {
  int playerNum;
  vector< Piece* > pieces;
};
\end{lstlisting}

  \begin{itemize}
  \item here, a forward declaration is obviously necessary
  \item Usually, declare the higher level class forward
  \end{itemize}
\end{frame}

\begin{frame}[fragile]
  \frametitle{Definition of member functions}

  \begin{onlyenv}<1-2>
\begin{lstlisting}[title=Player.hpp]
class Player {
  void doNextMove(Piece &piece, Event ev, string &msg);
  int getPlayerNum() { return playerNum; }
};
\end{lstlisting}

\begin{lstlisting}[title=Player.cpp]
void
Player::doNextMove(Piece &piece, Event ev, string &msg) {
  piece.tryStep(ev, msg);
}
\end{lstlisting}
  \end{onlyenv}
  \begin{onlyenv}<3-4>
\begin{lstlisting}[title=Player.hpp]
class Player {
  void doNextMove(Piece *piece, Event ev, string &msg);
  int getPlayerNum() { return playerNum; }
};
\end{lstlisting}

\begin{lstlisting}[title=Player.cpp]
void
Player::doNextMove(Piece *piece, Event ev, string &msg) {
  piece->tryStep(ev, msg);
}
\end{lstlisting}
  \end{onlyenv}

  \begin{onlyenv}<1>
    \begin{itemize}
    \item member functions are defined like normal functions
    \item a member function is referred by its name and class
    \item class name and member name are separated by ``::''
    \end{itemize}
  \end{onlyenv}

  \begin{onlyenv}<2>
    \begin{itemize}
    \item member functions are called by ``object.function()''
    \item object name and member name are separated by ``.''
    \item same holds for member variables
    \end{itemize}
  \end{onlyenv}

  \begin{onlyenv}<3>
    \begin{itemize}
    \item with a pointer to an object, you can use ``->'' like in C
    \item ``(*pointer).function()'' is the same as ``pointer->function()''
    \item almost true due to operator overloading (later)
    \end{itemize}
  \end{onlyenv}

  \begin{onlyenv}<4>
    \begin{itemize}
    \item class definitions are usually placed in header file ``\textit{class}.hpp''
    \item member function definitions in separate cpp-file ``\textit{class}.cpp''
    \item \emph{small} functions can be defined in the class definition (\alert{inlined})
    \end{itemize}
  \end{onlyenv}

\end{frame}

\begin{frame}[fragile]
  \frametitle{Referring to yourself: this}

\begin{lstlisting}
void Piece::register() {
  board.addPiece(this);
}
void Piece::escape() {
  board.removePiece(this);
}
\end{lstlisting}

  \begin{block}{}
    How can an object refer to ``itself''?\\
    \emph{\lstinline!Piece! registers itself with the board}
  \end{block}

  \begin{block}{Answer}
    In a member function, \lstinline!this! always points to the
    current object
  \end{block}

  \begin{itemize}
  \item \lstinline!this! is a pointer, although guaranteed to be valid
  \item there is no reference to the current object, use \lstinline!*this!
    if necessary
  \end{itemize}
\end{frame}

\begin{frame}[fragile,fragile]
  \frametitle{Creation and life time of objects}

  \begin{onlyenv}<1>
\begin{lstlisting}
int test() {
  for (int i = 0; i < 10; ++i) {
    Piece piece;
    piece.register();
    // here, piece is destroyed automatically
  }
}
\end{lstlisting}

    \begin{itemize}
    \item objects are created like variables in C by giving type and name
    \item ...and die at the end of the scope (code block)
    \item exception: static variables like in C
    \item therefore, after the loop above, not a single piece exists
    \end{itemize}
  \end{onlyenv}

  \begin{onlyenv}<2>
\begin{lstlisting}
int test() {
  for (int i = 0; i < 10; ++i) {
    Piece *piecePtr = new Piece;
    piecePtr->register();
    delete piecePtr;
    // without the older pieces remain, inaccessible
  }
}
\end{lstlisting}

    \begin{itemize}
    \item objects can also be created by \lstinline!new!
    \item these live till they are destroyed by explicitely calling
      \lstinline!delete! \emph{once}
    \item there should be \emph{one} responsible object for
      destroying, the \alert{owner}
    \item more than one owner leads to segmentation faults
    \item no owner (and therefore no \lstinline!delete!) to memory leaks
    \item in OOP, ownership frequently changes, making things
      difficult
    \end{itemize}
  \end{onlyenv}
\end{frame}

\begin{frame}[fragile]
  \frametitle{Constructor}

\begin{lstlisting}
class Game {
  Board* board;
  vector< Player* > players;
public:
  Game(int numPlayers = 2): board(0) {
    for (int i = 0; i < numPlayers; ++i)
      player.push_back(new Player);
  }
};
\end{lstlisting}

  \begin{itemize}
  \item constructor is a function initializing an object
  \item has the same name as the classs (``class::class'')
  \item can take arguments that specify what to create
  \item there can be several constructors (see overloading)
  \item initialize member variables after colon (here: \lstinline!board!)
  \item multiple variables separated by comma
  \end{itemize}
\end{frame}

\begin{frame}[fragile]
  \frametitle{Destructor}

\begin{lstlisting}
class Game {
  Board* board;
  vector< Player* > players;
public:
  ~Game() {
    for (auto player: players) delete player;
    delete board;
  }
};
\end{lstlisting}

  \begin{itemize}
  \item destructor is called right \emph{before} unallocating the
    memory
  \item should clean up
  \item in particular destroy all owned objects
  \item inform other objects that have pointers to this object
  \item the destructor cannot take arguments
  \end{itemize}
\end{frame}

\begin{frame}
  \frametitle{Default and copy constructor}



\end{frame}

\begin{frame}
  \frametitle{Function overloading}
  
  \begin{center}
    \begin{tikzpicture}
      \umlclass[type=abstract]{Tile}{ }{
        \umlvirt{getRepresentation(representation : string)}\\[\medskipamount]
        \umlvirt{tryStep(piece : Piece, how : Event, out message : string)}\\
        \umlvirt{tryStep(monster : Monster, how : Event, out message : string)}\\
      }
    \end{tikzpicture}
  \end{center}

  \begin{itemize}
  \item arguments with initializers
  \item a method is described by name and \alert{signature}
  \item signature is formed by the types of all taken arguments\\
    \emph{piece : Piece, how : Event, out message : string}\\
    \emph{monster : Monster, how : Event, out message : string}
  \item return value is \emph{not} part of the signature
  \end{itemize}

\end{frame}

\begin{frame}
  \frametitle{Operator overloading}
  
\end{frame}

\begin{frame}
  \frametitle{Inheritance}

  \begin{center}
    \begin{tikzpicture}
      \umlclass[y=2,x=4,type=abstract]{Tile}{
        game: Game
      }{
        \umlvirt{tryStep(...)}
      }
      \umlclass{FloorTile}{
      }{
        tryStep(...)
      }

      \umlclass[x=8]{WallTile}{
      }{
        tryStep(...)
      }

      \umlinherit{WallTile}{Tile}
      \umlinherit{FloorTile}{Tile}
    \end{tikzpicture}
  \end{center}
  
\end{frame}

\begin{frame}
  \frametitle{Encapsulation: private, public, protected}

  \tikzumlset{font=\tiny}
  \begin{center}
    \begin{tikzpicture}
      \umlclass[type=abstract]{Tile}{
        \# game: Game
      }{
        \umlvirt{+ tryStep(...)}
      }
      \umlclass[y=-2]{WallTile}{
      }{
        + tryStep(...)
      }
      \umlclass[x=4,y=-1]{Player}{
        - playerNum: int
      }{
      }
      
      \umlinherit{WallTile}{Tile}
      \umldep{Player}{WallTile}
      \umldep{Player}{Tile}
    \end{tikzpicture}
  \end{center}

\end{frame}

\begin{frame}
  \frametitle{Friends}
  
\end{frame}

\begin{frame}
  \frametitle{Polymorphism: virtual functions}
\end{frame}

\begin{frame}
  \frametitle{Pure virtual functions and abstract base classes}
  
\end{frame}

\begin{frame}
  \frametitle{Multiple inheritance}
  
\end{frame}

\begin{frame}
  \frametitle{The diamond problem}
  
\end{frame}

\end{document}
